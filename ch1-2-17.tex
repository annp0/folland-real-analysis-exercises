\section{Chapter 1: Measures}
\subsection*{Exercise 2}
Show that $\mathcal{B}_\mathbb{R}$ is generated by each of the following:
\par(a) the open intervals $\mathcal{E}_1=\{(a,b):a<b\}$,
\par(b) the closed intervals $\mathcal{E}_2=\{[a,b]:a<b\}$,
\par(c) the half-open intervals $\mathcal{E}_3=\{(a,b]:a<b)\}$ or $\epsilon_3=\{(a,b]:a<b)\}$,
\par(d) the open rays $\mathcal{E}_5=\{(a,\infty):a\in\mathbb{R}\}$ or $\mathcal{E}_6=\{(-\infty,a):a\in\mathbb{R}\}$,
\par(e) the closed rays $\mathcal{E}_7=\{[a,\infty)\:a\in\mathbb{R}\}$ or $\mathcal{E}_8=\{(-\infty,a]:a\in\mathbb{R}\}$.
\begin{proof}
    Recall lemma 1.1, which states if $\mathcal{E}\subset\mathcal{M}(\mathcal{F})$, then $\mathcal{M}(\mathcal{E})\subset\mathcal{M}(\mathcal{F})$. It is easy to observe $\mathcal{E}_i\subset\mathcal{B}_\mathbb{R}$, therefore $\mathcal{M}(\mathcal{E}_i)\subset\mathcal{B}_\mathbb{R}$.
    \par(a) Since every open set can be written as a countable union of intervals, denote $\mathcal{O}$ as the set of all open sets, then $\mathcal{O}\subset\mathcal{M}(\mathcal{E}_1)$, $\mathcal{B}_\mathbb{R}\subset\mathcal{M}(\mathcal{E}_1)$. Hence $\mathcal{B}_\mathbb{R}=\mathcal{M}(\mathcal{E}_1)$.
    \par(b) Attempt to show $\mathcal{E}_1\subset\mathcal{M}(\mathcal{E}_2)$. Apparently $(a,b)=\cup_{n=1}^{\infty}[a+1/n,b-1/n]$.
    \par(c) $\mathcal{E}_1\subset\mathcal{M}(\mathcal{E}_3)$ since $(a,b)=\cup_{n=1}^{\infty}(a,b-1/n]$. The same goes for $\mathcal{E}_4$.
    \par(d) $\mathcal{E}_3\subset\mathcal{M}(\mathcal{E}_5)$ since $(a,b]=(a,\infty)\cap((b,\infty))^c$. The same argument goes for $\mathcal{E}_6$.
    \par(e) $\mathcal{E}_4\subset\mathcal{M}(\mathcal{E}_7)$ since $[a,b)=[a,\infty]\cap([b,\infty))^c$.
    \par Therefore $\mathcal{B}_\mathbb{R}=\mathcal{M}(\mathcal{E}_i)$.
\end{proof}
\subsection*{Exercise 4}
An algebra $\mathcal{A}$ is a $\sigma$-algebra iff $\mathcal{A}$ is closed under countable increasing unions.
\begin{proof}
    If $\mathcal{A}$ is closed under countable increasing unions, for any countable collection of sets $\{F_j\}$ in $\mathcal{A}$, let $E_1=F_1$, $E_2=E_1\cup F_2$, $E_n=E_{n-1}\cup F_n$, then $\{E_j\}$ is an increasing sequence of sets, therefore $\cup^{\infty}_{n=1}E_n=\cup^{\infty}_{j=1}F_j\in\mathcal{A}$. Therefore $\mathcal{A}$ is a $\sigma$-algebra. The reverse is trivial.
\end{proof}
\subsection*{Exercise 5}
If $\mathcal{M}$ is the $\sigma$-algebra generated by $\mathcal{E}$, then $\mathcal{M}$ is the union of $\sigma$-algebras generated by $\mathcal{F}$ as $\mathcal{F}$ ranges over all countable subsets of $\mathcal{E}$.
\begin{proof}
    Let $A$ be the index set of all countable subsets of $\mathcal{E}$. First claim that $\mathcal{B}=\cup_{\alpha\in A}\mathcal{M}(\mathcal{F}_{\alpha})$ is a $\sigma$-algebra. $\forany E\in\mathcal{B}$, $E\in\mathcal{M}(\mathcal{F}_\alpha)$, therefore $E^c\in\mathcal{B}$. Given a countable collection of sets $\{E_j\}$ in $\mathcal{B}$, since $E_j\in\mathcal{M}({F_\alpha})$, $E_j$ must be in at least one $\mathcal{M}(\mathcal{F}_j)$. Let $\mathcal{H}=\cup^{\infty}_{j=1}\mathcal{F}_j$, consider $\mathcal{M}(\mathcal{H})$. Obviously $\{E_j\}\in\mathcal{M}(\mathcal{H})$, therefore $\cup^\infty_{j=1}E_j\in\mathcal{M}(\mathcal{H})$. Since $\mathcal{H}$ is also a countable subset of $\mathcal{E}$, $\mathcal{M}(\mathcal{H})\subset\mathcal{B}$. Therefore $\mathcal{B}$ is indeed a $\sigma$-algebra.
    \par It is straightforward that $\mathcal{E}\subset\mathcal{B}$. For the reverse, $\forany E\in\mathcal{B}$, $E$ is in some $\sigma$-algebra generated by $\mathcal{F}_\alpha$, therefore $E\in\mathcal{M}$. Thus $\mathcal{M}=\mathcal{B}$.
\end{proof}
\subsection*{Exercise 6}
Suppose that $(X,M,\mu)$ is a measure space. Let $\mathcal{N}=\{N\in M:\mu(N)=0\}$ and $\overline{\mathcal{M}}=\{E\cup F:E\in\mathcal{M}, F\subset N\in\mathcal{N}\}$. Then $\overline{\mathcal{M}}$ is a $\sigma$-algebra, and there is a unique extension $\overline{\mu}$ of $\mu$ to a complete measure on $\overline{\mathcal{M}}$.
\begin{proof}
    Apparently $\overline{\mathcal{M}}$ is closed under countable unions. For any $E\in\mathcal{M}, F\subset N\in\mathcal{N}$, without the loss of generality assume $E\cap N=\varnothing$ (otherwise replace $N,F$ with $N\backslash E$ and $F\backslash E$). Then $E\cup F=(E\cup N)\cap(N^c\cup F)$, $(E\cup F)^c=(E\cup N)^c\cup(N\cap F^c)^c\in\overline{\mathcal{M}}$. Therefore $\overline{\mathcal{M}}$ is a $\sigma$-algebra.
    \par Now consider the extension $\overline{\mu}$. Let $\overline{\mu}(E\cup F)=\mu(E)$. This is well-defined since if $E_1\cup F_1=E_2\cup F_2$ then $E_1\subset E_2\cup N_2$, $\mu(E_1)\le\mu(E_2)$, and likewise $\mu(E_1)\ge\mu(E_2)$, thus $\mu(E_1)=\mu(E_2)$. Then $\overline{\mu}(\varnothing)=\overline{\mu}(\varnothing\cup\varnothing)=0$, and the countable additivity can be likewise easily verified. For the uniqueness, give any other measure $\overline{\mu}'$, $\overline{\mu}'(E\cup F)\le\overline{\mu}'(E\cup N)\le\mu(E)$. But $\overline{\mu}'(E\cup F)\ge\overline{\mu}'(E\cup\varnothing)=\mu(E)$, thus $\overline{\mu}'=\overline{\mu}$.
\end{proof}
\subsection*{Exercise 7}
If $\mu_1,\dots,\mu_n$ are measures on $(X,\mathcal{M})$ and $a_1,\dots,a_n\in[0,\infty)$, then $\sum_1^n a_j\mu_j$ is a measure on $(X,\mathcal{M})$.
\begin{proof}
    Let $\mu'=\sum^n_1 a_j\mu_j$. Then $\mu'(\varnothing)=0$, given any collection disjoint sets $\{E_j\}$ in $\mathcal{M}$, $\mu'(\cup_1^\infty E_j)=\sum^n_1a_j\mu_j(\cup_1^\infty E_j)=\sum^{\infty}_{j=1}\sum^n_1a_j\mu_j(E_j)=\sum^\infty_{j=1}\mu'(E_j)$, therefore $\mu'$ is also a measure.
\end{proof}
\subsection*{Exercise 8}
If $(X,\mathcal{M},\mu)$ is a measure space and $\{E_j\}^\infty_1\subset\mathcal{M}$, then $\mu(\lim\inf E_j)\le\lim\inf\mu(E_j)$. Also, $\mu(\lim\sup E_j)\ge\lim\sup\mu(E_j)$ provided that $\mu(\cup^\infty_1E_j)<\infty$.
\begin{proof}
    Recall
    $$
    \lim\inf E_j=\bigcup_{j=1}^\infty\bigcap_{i=j}^\infty E_i,\quad \lim\sup E_j=\bigcap_{j=1}^\infty\bigcup_{i=j}^\infty E_i
    $$
    observe that $\{A_j=\cap^\infty_{i=j}E_j\}$ gives a sequence such that $A_1\subset A_2\cdots$, since $\mu$ is a measure, $\mu(\lim\inf E_j)=\mu(\cup^\infty_{j=1}A_j)=\lim_{j\to\infty}\mu(A_j)\le\lim\inf\mu(E_j)$. For the second claim, in the same sense let $\{B_j=\cup^\infty_{i=j}E_j\}$, then $\mu(\lim\sup E_j)=\lim_{j\to\infty}\mu(B_j)\ge\lim\sup\mu(B_j)$. 
\end{proof}
\subsection*{Exercise 9}
If $(X,\mathcal{M},\mu)$ is a measure space and $E,F\in\mathcal{M}$, then $\mu(E)+\mu(F)=\mu(E\cup F)+\mu(E\cap F)$.
\begin{proof}
    Since $\mu$ is a measure, $\mu(E)+\mu(F)=\mu(E\cap F)+\mu(E\cap F^c)+\mu(E\cap F)+\mu(E^c\cap F)=\mu(E\cup F)+\mu(E\cap F)$.
\end{proof}
\subsection*{Exercise 10}
Given a measure space $(X,\mathcal{M},\mu)$ and $E\in\mathcal{M}$, define $\mu_E(A)=\mu(A\cap E)$ for $A\in\mathcal{M}$. Then $\mu_E$ is also a measure.
\begin{proof}
    Apparently $\mu_E(\varnothing)=0$. Given any collection of disjoint sets $\{A_j\}$ in $\mathcal{M}$, $\mu_E(\cup^\infty_{j=1}A_j)=\mu(\cup^\infty_{j=1}A_j\cap E)=\mu(\cup^\infty_{j=1}(A_j\cap E))=\sum^\infty_{j=1}\mu_E(A_j)$. Therefore $\mu_E$ is a measure.
\end{proof}
\subsection*{Exercise 11}
A finitely additive measure $\mu$ is a measure iff it is continuous from below. If $\mu(X)<\infty$, $\mu$ is a measure iff it is countinuous from above.
\begin{proof}
    Given a finitely additive measure $\mu$, if it is continuous from below, then given a sequence of disjoint sets $\{E_j\}$, let $\{A_j=\cup_{i=1}^j E_i\}$, $\mu(\cup_jE_j)=\mu(\cup_jA_j)=\lim_{n\to\infty}\mu(A_n)$, by finite additivity $\lim_{n\to\infty}\mu(A_n)=\lim_{n\to\infty}\sum_{i=1}^n\mu(E_i)=\sum^\infty_{n=1}\mu(E_j)$. For the second claim, let $\{B_j=A_j^c\}$, then $\cup_j E_j=\cup_j A_j=(\cap_j(A_j^c))^c=(\cap_j(B_j))^c$, therefore $\mu(\cup_jE_j)+\mu(\cap_jB_j)=\mu(X)$. By continuity from above, $\mu(\cap_jB_j)=\lim_{j\to\infty}\mu(B_j)=\mu(X)-\lim_{j\to\infty}\mu(A_j)$, the rest is the same as the previous argument.
\end{proof}
\subsection*{Exercise 12}
Let $(X,\mathcal{M},\mu)$ be a finite measure space.
\par (a) If $E,F\in\mathcal{M}$ and $\mu(E\triangle F)=0$, then $\mu(E)=\mu(F)$.
\par (b) Say that $E\sim F$ if $\mu(E\triangle F)$. Then $\sim$ is an equivalence relation on $\mathcal{M}$.
\par (c) For $E,F\in M$, define $\rho(E,F)=\mu(E\triangle F)$. Then $\rho(E,G)\le\rho(E,F)+\rho(F,G)$, and hence $\rho$ defines a metric on the space $M/\sim$.
\begin{proof}
    Recall $E\triangle F=(E\backslash F)\cup(F\backslash E)$.
    \par (a) Since $E\backslash F$ and $F\backslash E$ are disjoint, $\mu(E\triangle F)=\mu(E\backslash F)+\mu(F\backslash E)$. Since $E=(E\backslash F)\cup(E\cap F)$, $F=(F\backslash E)\cup(E\cap F)$, $\mu(E\triangle F)=\mu(E)+\mu(F)-2\mu(E\cap F)$. Notice that $\mu(E)\ge\mu(E\cap F)$, $\mu(F)\ge\mu(E\cap F)$, therefore when $\mu(E\triangle F)=0$, $\mu(E)=\mu(F)=\mu(E\cap F)$.
    \par (b) Since ``$=$'' is an equivalence relation on $[0,\infty)$, ``$\sim$'' is obviously also an equivalence relation.
    \par (c) Attempt to verify $\mu(E\triangle G)\le\mu(E\triangle F)+\mu(F\triangle G)$:
    \begin{align*}
        \mu(E\triangle G)&=\mu(E\backslash G)+\mu(G\backslash E)\\
        &=\mu(E)+\mu(G)-2\mu(E\cap G)\\
        &\le\mu(E)+2\mu(F)+\mu(G)-2\mu(E\cap F)-2\mu(F\cap G)\\
        &=\mu(E\triangle F)+\mu(F\triangle G)
    \end{align*}
    where the inequality $\mu(F)+\mu(E\cap G)=\mu(F\cap E^c\cap G^c)+\mu(F\cap G^c\cap E)+\mu(F\cap G\cap E^c)+2\mu(F\cap G\cap E)+\mu(E\cap G\cap F^c)\ge\mu(F\cap E\cap G^c)+\mu(F\cap G\cap E^c)+2\mu(E\cap F\cap G)=\mu(E\cap F)+\mu(F\cap G)$ is utilized.
\end{proof}
\subsection*{Exercise 14}
If $\mu$ is a semifinite measure and $\mu(E)=\infty$, for any $C>0$ there exists $F\subset E$ with $C<\mu(F)<\infty$.
\begin{proof}
    Assume that there exists $C>0$ such that $\forany F\subset E$, $\mu(F)\le C$, then
    $\sup\{\mu(F):F\subset E\}\le C$. Denote the supremum with $S$. Then $\forany n\in\mathbb{N},\exist F_n\subset E$ such that $S-1/n<\mu(F_n)\le S$. Since $F'=\cup_{n=1}^\infty F_n\subset E$, $\mu(F')=S$. Then consider $E\backslash F'$. Obviously $\mu(E\backslash F')=\infty$. Because $\mu$ is semifinite, there exist $F''$ such that $0<\mu(F'')<\infty$. Then $\mu(F'\cup F'')>S$, contradiction. Therefore there is no supremum.
\end{proof}
\subsection*{Exercise 15}
Given a measure $\mu$ on $(X,\mathcal{M})$, define $\mu_0$ on $\mathcal{M}$ by $\mu_0(E)=\sup\{\mu(F):F\subset E\,\text{and}\,\mu(F)<\infty\}$.
\par (a) $\mu_0$ is a semifinite measure. It is called the semifinite part of $\mu$.
\par (b) If $\mu$ is semifinite, then $\mu=\mu_0$.
\par (c) There is a measure $\nu$ on $\mathcal{M}$ (in general, not unique) which assumes only values $0$ and $\infty$ such that $\mu=\mu_0+\nu$.
\begin{proof}
    (a) First verify that $\mu_0$ is a measure. Obviously $\mu_0(\varnothing)=0$. Give any collection of disjoint sets $\{E_j\}$, let $E=\cup_{j=1}^\infty E_j$. For a measurable set $F\subset E$ and $\mu(F)<\infty$, $\mu(F)=\sum_j\mu(F\cap E_j)\le\sum_j\mu_0(E_j)$. Since this holds for any subset of $E$ that has finite measure, $\mu_0(E)\le\sum_j\mu_0(E_j)$. If $\mu_0(E)=\infty$, then the reverse trivially holds. Otherwise $\mu_0(E)<\infty$. Then for each $E_j$, $\forany \epsilon/2^j$, there exists $F_j\subset E_j$ such that $\mu_0(E_j)-\epsilon/2^j<\mu(F_j)\le\mu_0(E_j)$. Then $\mu_0(E)\ge\mu(\cup_{j=1}^{\infty}F_j)=\sum_j\mu_0(E_j)-\epsilon$. Therefore $\mu_0(E)=\sum_j\mu_0(E_j)$, $\mu_0$ is a measure.
    
    Given a $E$ such that $\mu_0(E)=\infty$, take any $C>0$, then $\exists F\subset E$ such that $C<\mu(F)<\infty$. Then $\mu_0(F)=\mu(F)$ is non-zero and finite. Therefore $\mu_0$ is a semifinite measure.
    \par (b) For any $E\in\mathcal{M}$, if $\mu(E)<\infty$, then $\mu(E)=\mu_0(E)$. If $\mu(E)=\infty$, then by Exercise 14 $\mu_0(E)=\infty$. Therefore $\mu=\mu_0$.
    \par (c) Let 
    $$
    \nu(E)=\begin{cases}
        0,\quad\text{if $E$ is $\sigma$-finite}\\
        \infty,\quad\text{otherwise}
    \end{cases}
    $$
    $\nu$ is a measure since the disjoint union of $\sigma$-finite sets is still a $\sigma$-finite set, and if there is a set that is not $\sigma$-finite in the collection the union will also not be $\sigma$-finite. Now verify $\mu(E)=\mu_0(E)+\nu(E)$. When $E$ is $\sigma$-finite, if $\mu(E)$ is finite, then the quality holds. If $\mu(E)$ is not finite, then by previous exercise $\mu_0(E)=\infty$, the quality still holds. If $E$ is not $\sigma$-finite, the quality holds trivially.
\end{proof}
\subsection*{Exercise 16}
Let $(X,\mathcal{M},\mu)$ be a measure space. A set $E\subset X$ is called locally measurable if for all $A\in\mathcal{M}$ such that $\mu(A)<\infty$, $E\cap A\in\mathcal{M}$. Let $\widetilde{\mathcal{M}}$ be the collection of all locally measurable sets. Clearly $\mathcal{M}\subset\widetilde{\mathcal{M}}$; if $\mathcal{M}=\widetilde{\mathcal{M}}$, then $\mu$ is called saturated.
\par(a) If $\mu$ is $\sigma$-finite, then $\mu$ is saturated.
\par(b) $\widetilde{\mathcal{M}}$ is a $\sigma$-algebra.
\par(c) Define $\widetilde{\mu}$ on $\widetilde{\mathcal{M}}$ by $\widetilde{\mu}=\mu(E)$ if $E\in\mathcal{M}$ and $\widetilde{\mu}(E)=\infty$ otherwise. Then $\widetilde{\mu}$ is a saturated measure on $\widetilde{\mathcal{M}}$, called the saturation of $\mu$.
\par(d) If $\mu$ is complete, so is $\overline{\mu}$.
\par(e) Suppose that $\mu$ is semifinite. For $E\in\widetilde{\mathcal{M}}$, define $\underline{\mu}(E)=\sup\{\mu(A):A\in\mathcal{M}\,\text{and}\,A\subset E\}$. Then $\underline{\mu}$ is a saturated measure on $\widetilde{\mathcal{M}}$ that extends $\mu$.
\par(f) Let $X_1, X_2$ be disjoint uncountable sets, $X=X_1\cup X_2$, and $\mathcal{M}$ the $\sigma$-algebra of countable or co-countable sets in $X$. Let $\mu_0$ be counting measure on $\mathcal{P}(X_1)$ and define $\mu$ on $\mathcal{M}$ by $\mu(E)=\mu_0(E\cap X_1)$. Then $\mu$ is a measure on $\mathcal{M}$, $\widetilde{M}=\mathcal{P}(X)$, and in the notation of (c) and (e), $\widetilde{\mu}\neq\underline{\mu}$.
\begin{proof}
    (a) Since $\mu$ is $\sigma$-finite, there exists a countable collection of disjoint sets $\{E_j\}$ such that $X=\cup^\infty_{j=1}E_j$ and $\mu(E_j)\le\infty$. Therefore $\forany E\in\widetilde{\mathcal{M}}$, for each $E_j$, $E\cap E_j\in\mathcal{M}$. Thus $E=\cup_1^\infty(E\cap E_j)\in\mathcal{M}$. Hence $\widetilde{\mathcal{M}}=\mathcal{M}$.
    \par(b) $\forany E\in\widetilde{\mathcal{M}}$, $\forany A\in\mathcal{M}$ such that $\mu(A)<\infty$, $E^c\cap A=(A\cap(E\cap A)^c)\in\mathcal{M}$, therefore $E^c\in\widetilde{\mathcal{M}}$. Give any countable collection of sets $\{E_j\}$ in $\widetilde{\mathcal{M}}$, for any $A\in\mathcal{M}$ that has finite measure, $(\cup_j E_j)\cap A=\cup_j(E_j\cap A)\in\mathcal{M}$. Thus $\widetilde{\mathcal{M}}$ is a $\sigma$-algebra.
    \par(c) First check that $\widetilde{\mu}$ is a measure. Apparently $\widetilde{\mu}(\varnothing)=0$. Given any countable collection of disjoint sets $\{E_j\}$ in $\widetilde{\mathcal{M}}$, if $E_j\in\mathcal{M}$ for each $j$, then the additivity trivially holds. If $\exist i$ such that $E_i\not\in\mathcal{M}$, assume $\cup_jE_j\in\mathcal{M}$. Obviously $\cup_jE_j$ cannot have finite measure. Therefore the equality still holds. Then check that $\widetilde{\mu}$ is saturated. $\forany E$, if $\forany A\in\widetilde{\mathcal{M}}$ such that $\widetilde{\mu}(A)<\infty$, $A\cap E\in\widetilde{\mathcal{M}}$, then $\mu(A)<\infty$, therefore $\widetilde{\mu}(A\cap E)<\infty$, $A\cap E\in\mathcal{M}$, $E\in\widetilde{\mathcal{M}}$.
    \par(d) $\forany N\in\widetilde{\mathcal{M}}$, if $\widetilde{\mu}(N)=0$, then because $\mu$ is complete, $\forany F\subset N$, $\widetilde{\mu}(F)=0$. Therefore $\widetilde{\mu}$ is also complete.
    \par(e) First verify $\underline{\mu}$ is a measure. Obviously $\underline{\mu}(\varnothing)=0$. Given any countable collection of disjoint sets $\{E_j\}$ in $\widetilde{\mathcal{M}}$, assume they are all finite. $\forany E_j$, $\exist A_j$ such that $A_j\in\mathcal{M}$, $A_j\subset E_j$, $\underline{\mu}(E_j)-\epsilon/2^j<\mu(A_j)\le\underline{\mu}(E_j)$. Then $\underline{\mu}(\cup_jE_j)\ge\mu(\cup_jA_j)>\sum_j\underline{\mu}(E_j)-\epsilon$, therefore $\underline{\mu}(\cup_jE_j)\ge\sum_j\underline{\mu}(E_j)$. For the reverse inequality, take $A\in\mathcal{M}$ such that $\underline{\mu}(\cup_jE_j)-\epsilon<\mu(A)\le\underline{\mu}(\cup_jE_j)$, since $A\subset \cup_jE_j$ and $\mu(A)<\infty$, $A_j=A\cap E_j\in\mathcal{M}$, therefore $\underline{\mu}(\cup_jE_j)-\epsilon<\mu(A)\le\sum_j\underline{\mu}(E_j)$. Therefore the reverse inequality holds, $\underline{\mu}(\cup_jE_j)=\sum_j\underline{\mu}(E_j)$. For the infinite case, since $\mu$ is semifinite, by exercise 14 both inequality hold trivially. If $E\in\mathcal{M}$, then $\mu(E)=\underline{\mu}(E)$ since $\mu$ is semifinite. Therefore $\underline{\mu}$ is an extend of $\mu$.
    \par Now check that $\underline{\mu}$ is saturated. $\forany E\in\widetilde{\widetilde{\mathcal{M}}}$, $\forany A\in{\mathcal{M}}$ such that $\underline{\mu}(A)<\infty$, $E\cap A\in\widetilde{\mathcal{M}}$. Then $E\cap A=E\cap A\cap A\in\mathcal{M}$.
    \par(f) Since $\mu_0$ is a well-defined measure, it is straightforward that $\mu$ is also a measure. $\forany A\subset X$, given any $B$ such that $B\in\mathcal{M}$ and $\mu(B)<\infty$, since $B\cap X_1$ is finite, $B$ must be countable. Therefore $B\cap A$ is also countable, $B\cap A\subset\widetilde{\mathcal{M}}$. Therefore $\widetilde{\mathcal{M}}=\mathcal{P}(X)$. Obviously $\widetilde{\mu}\neq\underline{\mu}$, one example may be $\{x_1\}\cup X_2$ where $x_1\in X_1$.
\end{proof}
\subsection*{Exercise 17}
If $\mu^*$ is an outer measure on $X$ and $\{A_j\}_1^\infty$ is a sequence of disjoint $\mu^*$-measurable sets, then $\mu^*(E\cap(\cup^\infty_1A_j))=\sum^\infty_1\mu^*(E\cap A_j)$ for any $E\subset X$.
\begin{proof}
    Since $\cup_j(E\cap A_j)=E\cap(\cup_jA_j)$, $\mu^*(E\cap(\cup^\infty_1A_j))\le\sum^\infty_1\mu^*(E\cap A_j)$. For the reverse inequality, let $B_n=\cup_{i=1}^nA_i$. Then $\mu^*(E\cap B_n)=\mu^*(E\cap A_n)+\mu^*(E\cap B_n\cap A_n^c)=\mu^*(E\cap A_n)+\mu^*(E\cap B_{n-1})=\sum_{i=1}^n\mu^*(E\cap A_i)$. Since $\mu^*(E\cap B_\infty)\ge\mu^*(E\cap B_n)=\sum_1^n\mu^*(E\cap A_i)$ for any $n$, $\mu^*(E\cap(\cup^\infty_1A_j))\ge\sum^\infty_1\mu^*(E\cap A_j)$.
\end{proof}
