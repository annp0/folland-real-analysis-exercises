\paragraph{Exercise 45}
If $(X_j,\mathcal{M}_j)$ is a measurable space for $j=1,2,3$, then $\bigotimes^3_1\mathcal{M}_j=(\mathcal{M}_1\otimes\mathcal{M}_2)\otimes\mathcal{M}_3$. Moreover, if $\mu_j$ is a $\sigma$-finite measure on $(X_j,\mathcal{M}_j)$, then $\mu_1\times\mu_2\times\mu_3=(\mu_1\times\mu_2)\times\mu_3$
\begin{proof}
    $(\mathcal{M}_1\otimes\mathcal{M}_2)\otimes\mathcal{M}_3$ is generated by $\mathcal{E}=\{(E_1\times E_2)\times E_3:E_j\in\mathcal{M}_j\}$. By the natural identification, one takes $(X_1\times X_2)\times X_3=X_1\times X_2\times X_3$. Thus
    $\mathcal{E}=\{E_1\times E_2\times E_3:E_j\in\mathcal{M}_j\}$, which generates $\bigotimes^3_1\mathcal{M}_j$.
    \par Suppose $\mu_1,\mu_2,\mu_3$ are $\sigma$-finite. Then on the algebra $\mathcal{A}$ of rectangles,
    $$
    (\mu_1\times\mu_2)\times\mu_3((E_1\times E_2)\times E_3)=\mu_1(E_1)\mu_2(E_2)\mu_3(E_3)=\mu_1\times\mu_2\times\mu_3(E_1\times E_2\times E_3)
    $$
    since $(\mu_1\times\mu_2)\times\mu_3$ and $\mu_1\times\mu_2\times\mu_3$ are both $\sigma$-finite measures and they agree on $\mathcal{A}$, they are equal by the uniqueness assertion in theorem 1.14.
\end{proof}
\paragraph{Exercise 46}
Let $X=Y=[0,1]$, $\mathcal{M}=\mathcal{N}=\mathcal{B}_{[0,1]}$, $\mu$ is the Lebesgue measure, and $\nu$ is the counting measure. If $D=\{(x,x):x\in[0,1]\}$ is the diagonal in $X\times Y$, then $\int\int\chi_D\mathrm{d}\mu\mathrm{d}\nu$, $\int\int\chi_D\mathrm{d}\nu\mathrm{d}\mu$, and $\int\chi_D\mathrm{d}(\mu\times\nu)$ are all unequal.
\begin{proof}
    Obviously,
    $$
    \int\int\chi_D\mathrm{d}\mu\mathrm{d}\nu=\int\left[\int\chi_D^y\mathrm{d}\mu\right]\mathrm{d}\nu=0
    $$
    $$
    \int\int\chi_D\mathrm{d}\nu\mathrm{d}\mu=\int\left[\int\chi_D^x\mathrm{d}\nu\right]\mathrm{d}\mu=\int\mathrm{d}\mu=1
    $$
    By definition,
    $$
    \int\chi_D\mathrm{d}(\mu\times\nu)=\inf\{\sum_{n=1}^\infty\mu(A_j)\nu(B_j):D\subset\cup_j(A_j\times B_j)\,\text{where $A_j\times B_j$ are disjoint rectangles}\}
    $$
    Suppose such sequence $A_j\times B_j$ that covers $D$. Then $[0,1]\subset\cup_j(A_j\cap B_j)$. Therefore $\mu(A_n\cap B_n)>0$ for some $n$. Then $\mu(A_n)>0$, and $\nu(B_n)=\infty$. Therefore the integral is $\infty$.
\end{proof}
\paragraph{Exercise 48}
Let $X=Y=\mathbb{N}$, $\mathcal{M}=\mathcal{N}=\mathcal{P}(\mathbb{N})$, $\mu$ and $\nu$ are the counting measure. Define $f(m,n)=1$ if $m=n$ and $f(m,n)=-1$ if $m=n+1$, and $f(m,n)=0$ otherwise. Then $\int|f|\mathrm{d}(\mu\times\nu)=\infty$, and $\int\int f\mathrm{d}\mu\mathrm{d}\nu$ and $\int\int f\mathrm{d}\nu\mathrm{d}\mu$ exist and are unequal.
\begin{proof}
    $$
    \int\left[\int f^y\mathrm{d}\mu\right]\mathrm{d}\nu=\sum^{\infty}_{n=0}\sum^{\infty}_{j=0}f(n,j)=0
    $$
    $$
    \int\left[\int f_x\mathrm{d}\nu\right]\mathrm{d}\mu=\sum^{\infty}_{j=0}\sum^{\infty}_{n=0}f(n,j)=1
    $$
    Let $E_1=\{(n,n):n\in\mathbb{N}\}$ and $E_2=\{(n,n+1):n\in\mathbb{N}\}$, then $|f(x)|=1$ and non-zero iff $x\in E_1\cup E_2$. Thus
    $$
    \int |f|\mathrm{d}(\mu\times\nu)=(\mu\times\nu)(E_1)+(\mu\times\nu)(E_2)=\infty
    $$
    since $E_1$ and $E_2$ are not finite.
\end{proof}
\paragraph{Exercise 49}
Prove Theorem 2.39 by using Theorem 2.37 and Proposition 2.12 together with the following lemmas.
\par (a) If $E\in\mathcal{M}\otimes\mathcal{N}$ and $\mu\times\nu(E)=0$, then $\nu(E_x)=\mu(E^y)=0$ for a.e. $x$ and $y$.
\par (b) If $f$ is $\mathcal{L}$-measurable and $f=0$ $\lambda$-a.e., then $f_x$ and $f^y$ are integrable for a.e. $x$ and $y$, and $\int f_x\mathrm{d}\nu=\int f^y\mathrm{d}\mu=0$ for a.e. $x$ and $y$.
\begin{proof}
    \par (a) Since $\mu$ and $\nu$ are $\sigma$-finite,
    $$
    0=(\mu\times\nu)(E)=\int\mu(E^y)\mathrm{d}\nu(y)=\int\nu(E_x)\mathrm{d}\mu(x)
    $$
    therefore $\nu(E_x)=\mu(E^y)=0$ a.e. $x$ and $y$.
    \par (b) Let $E\subset X\times Y$ be the null set such that $f(x,y)=0$ for all $(x,y)\not\in E$. Since $\lambda$ is the completion of $\mu\times\nu$, there is a set $E'\in\mathcal{M}\otimes\mathcal{N}$ such that $E\subset E'$ and $(\mu\times\nu)(E')=0$. Therefore
    $$
    0=(\mu\times\nu)(E')=\int\mu(E'^y)\mathrm{d}\nu(y)=\int\nu(E_x')\mathrm{d}\mu(x)
    $$
    thus $\nu(E'_x)=0$ and $\mu(E'^y)=0$ a.e. Since $\mu$ and $\nu$ are complete, $\mu(E_x)=0$ and $\nu(E^y)=0$ a.e. Therefore $f_x=0$ and $f^y=0$ a.e. Hence $f_x$ and $f^y$ are measurable and integrable a.e. with $\int f_x\mathrm{d}\nu=\int f^y\mathrm{d}\mu=0$.
    \par Now assume $f$ is $\mathcal{L}$-measurable. There exists an $(\mathcal{M}\otimes N)$-measurable function $g$ such that $f=g$ $\lambda$-a.e. If $f\ge 0$, then $g\ge 0$ a.e. Without the loss of generality assume $g\ge 0$, by Tonelli's theorem, $x\mapsto\int g_x\mathrm{d}\nu$ and $y\mapsto\int g^y\mathrm{d}\mu$ are non-negative and $(M\otimes N)$-measurable with
    $$
    \int g\mathrm{d}\lambda=\int\int g(x,y)\mathrm{d}\mu(x)\mathrm{d}\nu(y)=\int\int g(x,y)\mathrm{d}\nu(y)\mathrm{d}\mu(x)\quad (*)
    $$
    Since $g=f$ $\lambda$-a.e., if $f\in L^1(\lambda)$ then $g\in L^1(\mu\times\nu)$. By Fubini's theorem, this implies that $g_x\in L^1(\nu)$, $g_y\in L^1(\mu)$, $x\mapsto\int g_x\mathrm{d}\nu\in L^1(\mu)$ and $y\mapsto g_y\mathrm{d}(\mu)\in L^1(\nu)$ a.e. $x$ and $y$,  and $(*)$ holds.
    \par Apply (b) to $f-g$, therefore $f_x\in L^1(\nu)$ and $f^y\in L^1(\mu)$ a.e. $x$ and $y$ provided that $f\in L^1(\lambda)$. In either cases, $\int g_x\mathrm{d}\nu=\int f_x\mathrm{d}\nu$ a.e. $x$, therefore $\int f_x\mathrm{d}\nu$ is measurable and the same holds for $y$. Because $f=g$ a.e.,
    \begin{align*}
        \int f\mathrm{d}\lambda&=\int g\mathrm{d}\lambda\\
        &=\int\int g(x,y)\mathrm{d}\mu(x)\mathrm{d}\nu(y)=\int\int g(x,y)\mathrm{d}\nu(y)\mathrm{d}\mu(x)\\
        &=\int\int f(x,y)\mathrm{d}\mu(x)\mathrm{d}\nu(y)=\int\int f(x,y)\mathrm{d}\nu(y)\mathrm{d}\mu(x)
    \end{align*}
\end{proof}
\paragraph{Exercise 50}
Suppose $(X,\mathcal{M},\mu)$ is a $\sigma$-finite measure space and $f\in L^+(X)$. Let 
$$
G_f=\{(x,y)\in X\times[0,\infty]:y\le f(x)\}
$$
then $G_f$ is $\mathcal{M}\times\mathcal{B}_{\mathbb{R}}$-measurable and $\mu\times m(G_f)=\int f\mathrm{d}\mu$; the same is also true if the inequality $y\le f(x)$ in the definition of $G_f$ is replaced by $y<f(x)$.
\begin{proof}
    Since $g=(x,y)\mapsto(f(x)-y)=((s,t)\mapsto (s-t))\circ((x,y)\mapsto(f(x),y))$, $G_f=g^{-1}([0,\infty))$ is measurable. Then
    $$
    (\mu\times m)(G_f)=\int m((G_f)_x)\mathrm{d}\mu(x)=\int f\mathrm{d}\mu
    $$
\end{proof}
\paragraph{Exercise 51}
Let $(X,\mathcal{M},\mu)$ and $(Y,\mathcal{N},\nu)$ be arbitrary measure spaces.
\par (a) If $f:X\to\mathbb{C}$ is $\mathcal{M}$-measurable, $g:Y\to\mathbb{C}$ is $\mathcal{N}$-measurable, and $h(x,y)=f(x)g(y)$, then $h$ is $\mathcal{M}\otimes\mathcal{N}$-measurable.
\par (b) If $f\in L^1(\mu)$ and $g\in L^1(\nu)$, then $h\in L^1(\mu\times\nu)$ and $\int h\mathrm{d}(\mu\times\nu)=(\int f\mathrm{d}\mu)(\int g\mathrm{d}\nu)$
\begin{proof}
    \par (a) Since $f(x)$ and $g(y)$ are $\mathcal{M}\otimes\mathcal{N}$-measurable, $h=fg$ is also measurable.
    \par (b) Suppose $f\ge 0$ and $g\ge 0$. Then there exist increasing sequences $\phi_n$ and $\psi_n$ of non-negative simple functions that converges to $f$ and $g$ respectively. Then $\phi_n\psi_n\to h$ pointwise. Suppose $\phi_n=\sum^k_ia_i\chi_{A_i}$, $\psi_n=\sum^l_jb_j\chi_{B_j}$. Then
    $$
    \int\phi_n\psi_n=\sum^k_i\sum^l_ja_ib_j(\mu\times\nu)(A_i\times B_j)=\left(\sum^k_ia_i\mu(A_i)\right)\left(\sum^l_jb_j\nu(B_j)\right)=\int\phi_n\cdot\int\psi_n
    $$
    therefore it is true for positive functions. For any complex function $g$, just decompose it into $u=\mathrm{Re}g$, $v=\mathrm{Im}g$ then $u^+$, $u^-$, $v^+$, $v^-$. Apply the above formula repeatedly, the proof is complete.
\end{proof}
\paragraph{Exercise 52}
The Fubini-Tonelli theorem is valid when $(X,\mathcal{M},\mu)$ is an arbitrary measure space and $Y$ is a countable sets, $\mathcal{N}=\mathcal{P}(Y)$, and $\nu$ is counting measure on $Y$.
\begin{proof}
    If $f\in L^+(X\times Y)$, since $\nu$ is the counting measure, identify it with $\mathbb{N}$. Then
    $$
    \int_X\int_{\mathbb{N}}f_x(n)\mathrm{d}\nu\mathrm{d}\mu=\int_X\left(\sum^\infty_1f_x(n)\right)\mathrm{d}\mu=\int_{\mathbb{N}}\int_Xf^n(x)\mathrm{d}\mu\mathrm{d}\nu=\sum^\infty_1\left(\int_Xf^n(x)\mathrm{d}\mu\right)=\int f\mathrm{d}(\mu\times\nu)
    $$
    therefore Fubini-Tonelli theorem is true.
\end{proof}