\section{Chapter 3: Signed Measures and Differentiation}
\paragraph*{Exercise 1}
Prove Proposition 3.1.
\begin{proof}
    Suppose $\{E_j\}$ an increasing sequence, $F_j=E_j\backslash\cup_1^{j-1}E_i$, since $\mu(E_j)=\sum_{k=1}^{n}\mu(F_k)$,
    $$
    \mu(\cup_jE_j)=\mu(\cup_jF_j)=\sum_j\mu(F_j)=\lim\mu(E_j)
    $$
    Suppose $\{E_j\}$ an decreasing sequence, since $\mu(E_1)<\infty$,
    $$
    \mu(\cap_jE_j)=\mu(E_1\backslash(E_1\backslash\cap_jE_j))=\mu(E_1)-\mu(\cup_j(E_1\backslash E_j))=\mu(E_1)-\lim(\mu(E_1)-\mu(E_j))=\lim\mu(E_j)
    $$
\end{proof}
\paragraph*{Exercise 3}
Let $\nu$ be a signed measure on $(X,\mathcal{M})$.
\par (a) $L^1(\nu)=L^1(|\nu|)$
\par (b) If $f\in L^1(\nu)$, $|\int fd\nu|\le\int|f|d|\nu|$
\par (c) If $E\in\mathcal{M}$, $|\nu|(E)=\sup\{|\int_Efd\nu|:|f|\le 1\}$.
\begin{proof}
    (a) Let $\phi\in L^1$ be a simple function, and write $\phi=\sum^n_{i=1}a_i\chi_{E_i}$, then
    $$
    \int\phi d|\nu|=\sum^n_{i=1}a_i|\nu|(E_i)=\sum^n_{i=1}a_i(\nu^+(E_i)+\nu^-(E_i))=\int\phi d\nu^++\int\phi d\nu^-
    $$
    since for any $f\in L^1(\nu)$, $f\in L^1(\nu^+)\cap L^1(\nu^-)$, thus
    $$
    \int|f|d|v|=\left\{\int\phi d|\nu|:\phi\in L^+\text{ simple, }\phi\le|f|\right\}=\int|f|dv^++\int|f|dv^-\le\infty
    $$
    hence $L^1(\nu)\subset L^1(|\nu|)$. The converse is obviously true.
    \par (b)
    $$
    \left|\int fd\nu\right|=\left|\int fd\nu^+-\int fd\nu^-\right|=\left|\int fd\nu^+-\int fd\nu^-\right|\le\int|f|dv^++\int|f|dv^-=\int|f|d|\nu|
    $$
    \par (c) Suppose $g=\chi_B-\chi_A$, where $A$ and $B$ are the hahn decomposition of $\nu$. Then
    $$
    \int_Egd\nu=\int(\chi_B-\chi_A)\chi_Ed\nu=\int\chi_{B\cap E}d\nu^++\int\chi_{A\cap E}\nu^-=\nu^+(E)+\nu^-(E)=|\nu|(E)
    $$
    If $|\nu|(E)=\infty$, the proof is done. Otherwise assume that $|\nu|(E)<\infty$, and let $f$ be a measurable function with $|f|\le 1$. Then $|\int_Efd\nu|\le\int_E|f|d|\nu|\le|\nu|(E)$. Therefore
    $$
    |\nu|(E)\le\left\{|\int_Efd\nu|:|f|\le 1\right\}\le|\nu|(E)
    $$
    the proof is complete.
\end{proof}
\paragraph*{Exercise 4}
If $\nu$ is a signed measure and $\lambda,\mu$ are positive measures such that $\nu=\lambda-\mu$, then $\lambda\ge\nu^+$ and $\mu\ge\nu^-$.
\begin{proof}
    Suppose hahn decomposition $A,B$ for $\nu$, then $\forany E\in\mathcal{M}$,
    $$
    \lambda(E)\ge\lambda(E\cap B)\ge\nu(E\cap B)=\nu^+(E\cap B)\ge\nu(E)
    $$
    the same argument goes for $\mu\ge\nu^-$.
\end{proof}
\paragraph*{Exercise 5}
If $\nu_1$, $\nu_2$ are signed measures that both omit the value $+\infty$ or $-\infty$, then $|\nu_1+\nu_2|\le|\nu_1|+|\nu_2|$.
\begin{proof}
    Obviously $\nu_1+\nu_2$ is still a signed measure, and $\nu_1+\nu_2=(\nu_1^++\nu_2^+)-(\nu_1^-+\nu_2^-)$. By exercise 4, $(\nu_1^++\nu_2^+)\ge(\nu_1+\nu_2)^+$ and $(\nu_1^-+\nu_2^-)\ge(\nu_1+\nu_2)^-$. Therefore
    $$
    |\nu_1+\nu_2|=(\nu_1+\nu_2)^++(\nu_1+\nu_2)^-\le (\nu_1^++\nu_2^+)+(\nu_1^-+\nu_2^-)=|\nu_1|+|\nu_2|
    $$
\end{proof}
\paragraph*{Exercise 6}
Suppose $\nu(E)=\int_E fd\mu$ where $\mu$ is a positive measure and $f$ is an extended $\mu$-integrable function. Describe the Hahn decompositions of $\nu$ and the positive, negative, and total variations of $\nu$ in terms of $f$ and $\mu$.
\begin{solution}
    $P=\{x:f(x)\ge 0\}$, $N=\{x:f(x)<0\}$. $\nu^+=\int_{E\cap P}fd\nu$, $\nu^-=-\int_{E\cap N}fd\nu$, $|\nu|=\nu^++\nu^-$.
\end{solution}
\paragraph*{Exercise 7}
Suppose that $\nu$ is a signed measure on $(X,\mathcal{M})$ and $E\in\mathcal{M}$.
\par (a) $\nu^+(E)=\sup\{\nu(F):F\in\mathcal{M},F\subset E\}$ and $\nu^-(E)=-\inf\{\nu(F):F\in\mathcal{M},F\subset E\}$.
\par (b) $|\nu|(E)=\sup\{\sum^n_1|\nu(E_j)|:n\in\mathbb{N},\text{$E_1,\cdots,E_n$ are disjoint, and $\cup^n_1E_j=E$}\}$.
\begin{proof}
    (a) Let $A$ and $B$ be the hahn decomposition. Then
    $$
    \nu^+(E)=\nu^+(E\cap P)\le\sup\{\nu(F):F\subset E\}
    $$
    moreover, if $F\subset E$, then
    $$
    \nu(F)=\nu^+(F)\le\nu^+(E)
    $$
    therefore
    $$
    \nu^+(E)=\sup\{\nu(F):F\subset E\}
    $$
    the similar argument works for $v^-(E)$.
    \par (b) Denote RHS with $t$.
    $$
    |\nu|(E)=|\nu(E\cap A)|+|\nu(E\cap B)|\le t
    $$
    moreover, 
    $$
    \sum^n_1|\nu(E_j)|\le\sum^n_1(\nu^+(E_j)+\nu^-(E_j))=\nu^+(E)+\nu^-(E)=|\nu|(E)
    $$
    the proof is complete.
\end{proof}
\paragraph*{Exercise 8}
$\nu\ll\mu$ iff $|\nu|\ll\mu$ iff $\nu^+\ll\mu$ and $\nu^-\ll\mu$.
\begin{proof}
    Suppose $\mu(E)=0$, then $|\nu|(E)=0$ iff $\nu^+(E)=\nu^-(E)=0$. If $\nu\ll\mu$, since $\forany F\in\mathcal{M}$ that is contained in $E$, $\nu(F)=\mu(F)=0$, by exercise 2, $|\nu|(E)=0$. The converse is trivial.
\end{proof}
\paragraph*{Exercise 9}
Suppose $\{v_j\}$ is a sequence of positive measures. If $\nu_j\perp\mu$ for all $j$, then $\sum^\infty_1\nu_j\perp\mu$; and if $\nu_j\ll\mu$ for all $j$, then $\sum^\infty_1\nu_j\ll\mu$.
\begin{proof}
    The second part is trivial by countable additivity. For the first part, denote $E_j$ the $\nu_j$-null set and $E_j^c$ the $\mu$-null set, then $\cap_jE_j$ is $\sum^\infty_1\nu_j$-null and $(\cap_jE_j)^c$ is $\mu$-null.
\end{proof}
\paragraph*{Exercise 10}
Theorem 3.5 may fail when $\nu$ is not finite.
\begin{solution}
    Take $d\nu(x)=dx/x$ and $d\mu(x)=dx$ on $(0,1)$. Then obviously $\nu\ll\mu$, but consider $E_n=(0,1/n)$, obviously $\nu(E_n)>1$.
\end{solution}
\paragraph*{Exercise 11}
Let $\mu$ be a positive measure. A collection of functions $\{f_\alpha\}_{\alpha\in A}\subset L^1(\mu)$ is called uniformly integrable if for every $\epsilon>0$ there exists $\delta>0$ such that $|\int_Ef_\alpha d\mu|<\epsilon$ for all $\alpha\in A$ whenever $\mu(E)<\delta$.
\par (a) Any finite subset of $L^1(\mu)$ is uniformly integrable.
\par (b) If $\{f_n\}$ is a sequence in $L^1(\mu)$ that converges in the $L^1$ metric to $f\in L^1(\mu)$, then $\{f_n\}$ is uniformly integrable.
\begin{proof}
    (a) Since $f\in L^1(\mu)$, the finite signed measure $E\mapsto\int_E fd\mu$ is absolutely continuous with respect to $\mu$. Therefore for any $\epsilon>0$, $\exist\delta_\alpha$ such that $|\int_Ef_\alpha d\mu|<\epsilon$ when $\mu(E)<\delta_\alpha$. Just take $\delta=\min\{\delta_\alpha\}>0$.
    \par (b) For any $\epsilon>0$, there exists $N\in\mathbb{N}$ such that $\int|f_n-f|d\mu<\epsilon/2$ for any $n\ge N$. Let $I=\{0,1,2,\cdots,N\}$ (with $f_0=f$), then $\{f_i\}_{i\in I}$ is uniformly integrable. Therefore $\exist \delta>0$ such that $|\int_Ef_id\mu|<\epsilon/2$ for any $i\in I$ with $\mu(E)<\delta$. Then for $i\in\mathbb{N}\backslash I$,
    $$
    \left|\int_Ef_nd\mu\right|=\left|\int_E(f_n-f)d\mu+\int_Efd\mu\right|\le\left|\int_E(f_n-f)d\mu\right|+\left|\int_E fd\mu\right|\le\epsilon
    $$
\end{proof}
\paragraph*{Exercise 12}
For $j=1,2$, let $\mu_j,\nu_j$ be $\sigma$-finite measures on $(X_j,\mathcal{M}_j)$ such that $\nu_j\ll\mu_j$. Then $\nu_1\times\nu_2\ll\mu_1\times\mu_2$, and
$$
\dfrac{d(\nu_1\times\nu_2)}{d(\mu_1\times\mu_2)}(x_1,x_2)=\dfrac{d\nu_1}{d\mu_1}(x_1)\dfrac{d\nu_2}{d\mu_2}(x_2)
$$
\begin{proof}
    If $(\mu_1\times\mu_2)(E)=0$, then
    $$
    0=\int\mu_2(E^{x_1})d\mu_1(x_1)
    $$
    therefore $\mu_2(E^{x_1})$ is $\mu_1$ a.e. and hence $\nu_2(E^{x_1})$ is $\nu_1$ a.e., then
    $$
    (\nu_1\times\nu_2)(E)=\int\nu_2(E^{x_1})d\nu_1(x_1)=0
    $$
    The second part is verified by
    \begin{align*}
        (\nu_1\times\nu_2)(E)&=\int f\chi_Ed(\mu_1\times\mu_2)=\int\left[\int f\chi_E d\mu_2(x_2)\right]d\mu_1(x_1)\\
        &=\int\nu_2(E^{x_1})d\nu_1(x_1)=\int\left[\int_E\dfrac{d\nu_2}{d\mu_2}(x_2)d\mu_2(x_2)\right]d\nu_1(x_1)\\
        &=\int\left[\int\chi_E\dfrac{d\nu_1}{d\mu_1}(x_1)\dfrac{d\nu_2}{d\mu_2}(x_2)d\mu_2(x_2)\right]d\mu_1(x_1)
    \end{align*}
\end{proof}
\paragraph*{Exercise 13}
Let $X=[0,1]$, $\mathcal{M}=\mathcal{B}_{[0,1]}$, $m$ is the Lebesgue measure, and $\mu$ is the counting measure on $\mathcal{M}$.
\par(a) $m\ll\mu$ but $dm\neq fd\mu$ for any $f$.
\par(b) $\mu$ has no Lebesgue decomposition with respect to $m$.
\begin{proof}
    (a) The first part is trivial. Suppose there exists such $f$, then $m(\{x\})=f(x)=0$, therefore $f=0$ and
    $$
    1=m(X)=\int_Xfd\mu=0
    $$
    contradiction.
    \par (b) Suppose that $\mu$ has a Lebesgue decomposition $\lambda+\rho$ with respect to $m$, with $\lambda\perp m$ and $\rho\ll m$. Then $\rho(\{x\})=0$, and $\lambda(\{x\})=1$. Suppose $X=A\sqcup B$ the Lebesgue decomposition with $\lambda(A)=m(B)=0$. Then $A=\varnothing$, then $m(B)=m(X)=1$, contradiction.
\end{proof}
\paragraph*{Exercise 16}
Suppose that $\mu,\nu$ are $\sigma$-finite measures on $(X,\mathcal{M})$ with $\nu\ll\mu$, and let $\lambda=\mu+\nu$. If $f=d\nu/d\lambda$, then $0\le f<1$ $\nu$-a.e. and $d\nu/d\mu=f/(1-f)$.
\begin{proof}
    Let $E_n=\{x:f(x)<-1/n\}$. Therefore
    $$
    -n^{-1}\lambda(E_n)\ge\int_{E_n}fd\lambda=\nu(E_n)\ge 0
    $$
    and hence $\mu(E_n)\le\lambda(E_n)=0$. It follows that $\mu(\cup^\infty_1E_n)=0$, so $f\ge 0$ $\mu$-a.e. Set $F=\{x:f(x)\ge 1\}$. Since $\nu$ is $\sigma$-finite, there is a sequence $F_n$ of subsets of $F$ which cover $F$ such that $\nu(F_n)<\infty$ for each $n$. Because
    $$
    \nu(F_n)=\int_{F_n}fd\lambda\ge\int_{F_n}1d\lambda=\lambda(F_n)=\mu(F_n)+\nu(F_n)
    $$
    $\mu(F_n)=0$. Thus $\mu(F)=0$ and $f<1$ $\mu$-a.e. Therefore $f,1-f\in L^+$, so for each $E\in\mathcal{M}$,
    $$
    \int_E(1-f)d\lambda+\nu(E)=\int_E1d\lambda=\lambda(E)=\mu(E)+\nu(E)
    $$
    Thus $\int_E(1-f)d\lambda=\mu(E)$ for any $\nu(E)<\infty$. This result extends to all $E\in\mathcal{M}$ since $\nu$ is $\sigma$-finite. Thus $d\mu/d\lambda=(1-f)$. Therefore
    $$
    \dfrac{d\nu}{d\mu}=\dfrac{d\nu}{d\lambda}\dfrac{d\lambda}{d\nu}=\dfrac{f}{1-f}
    $$
\end{proof}
\paragraph*{Exercise 17}
Let $(X,\mathcal{M},\mu)$ be a finite measure space, $\mathcal{N}$ a sub-$\sigma$-algebra of $\mathcal{M}$, and $\nu=\mu|\mathcal{N}$. If $f\in L^1(\mu)$, there exists $g\in L^1(\nu)$ such that $\int_Efd\mu=\int_Egd\nu$ for all $E\in\mathcal{N}$; if $g'$ is another such function then $g=g'$ $\nu$-a.e.
\begin{proof}
    Define $\lambda$ on $\mathcal{N}$ by $\lambda(E)=\int_Efd\mu$, since $\rho\ll\nu$, the rest is obvious by the Radon-Nikodym theorem.
\end{proof}